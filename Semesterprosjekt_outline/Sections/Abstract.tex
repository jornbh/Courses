\chapter{Introduction}


\section{Abstract}
Normally, when handling time-delays, Padé-approximations are used to estimate the delay. Although they asymptotically give the same response as a time-delay, there are no guarantees for how they will act in reality when they are interconnected. Discretising the system into the z-domain provides exact solutions, but the space is quite distorted and very dependent on the sampling rate. Analysis in the z-domain tends to be difficult as a result. This project report proposes the usage of the w-transform with to give estimates that may provide a more accurate analysis of robustness, and who provide stronger guarantees for the quality of the estimate.

The way this will be done will be to give exact solutions for all continuous parts by assuming that all outputs from the controllers are zero-order hold. After that, the exact solutions will be used together with the control laws for the internal states in the controller, as well as a state, representing the delayed input. The resulting closed-loop system will then be transformed into the w-domain, where it can be analysed. When analyzing the sensitivity to modeling errors, the eigenvalues of the closed-loop matrix is differentiated with respect to a physical parameter. Which means stepping back through the transformation by using the chain-rule. The expression can not be differentiated exactly, so an estimate is given instead, as well as an upper bound on the magnitude of the error. 

\section{Sammendrag}
\todo[inline]{Legg inn et sammendrag som sier det samme som abstractet}
Når man jobber med tidsforsinkelser tar man som oftest i bruk padé-approksomasjoner. Selv om de asymptotisk gir samme respons som en tids-forsinkelse er det ingen garantier for hvordan de vil fungere i praksis om flere er koblet sammen i et lineært system. Det er mulig å ta i bruk diskretisering, siden det gir eksakte løsninger, men z-domenet er ganske forvrengt, og veldig avhengig av samplings-rate. På grunn av dette kan det være vanskelig å bruke z-domenet til stabilitetsanalyse. I dette prosjektet foreslås det å bruke w-transformasjonen i stabilitetsanalysen for å få estimater som forhåpentilgvis er mer presise, og som også kan gi sterkere garantier for kvaliteten av estimatet. 